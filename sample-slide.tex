\documentclass[xetex]{beamer}

\mode<presentation>
{
	\usetheme[somber=false,ruled=true,arrow=true,square=false,shadow=true]{xelatexbengali}
}
\newcommand{\ignore}[1]{}

\begin{document}

\title[BeamerThemeXeLaTeXBengali]{জিলাটেক ‌ \XeLaTeX‌\ দিয়ে বাংলায় প্রেজেন্টেশন}
\author{নিউটন মু.আ.হাকিম}
\institute{mahnewton@gmail.com}

%\date{}


\begin{frame}{}
   \titlepage
\end{frame}

%\ignore{
\begin{frame}
	\frametitle{আলোচ্য সুচী}
	\begin{columns}
	\column{0.47\textwidth}
	\tableofcontents
	\column{0.47\textwidth}
	\end{columns}
\end{frame}

\section{পরিচিতি}
\begin{frame}{থিম পরিচিতি}
	\begin{block}{থিম}
	\begin{description}
		\item[নাম:] ‌{\rm xelatexbengali}; সব ছোট হাতের।
		\item[ব্যবহার:] {\rm sample-slide.tex} ফাইল দেখুন
		\item[অপশন:] {\rm somber,ruled,arrow,square,shadow}
		\item[অপশনের মান:] সবগুলো {\rm boolean; $\langle$option$\rangle$=true or false}
	\end{description}
	\end{block}
	\begin{block}{ইনস্টলেশন}
	sample-article.pdf বা readme-article.pdf এ জিলাটেকবেংগলি ইন্সটলেশন প্রসিডিউর দেখুন।
	\end{block}
\end{frame}

\section{দরকারী ফাইল}

\begin{frame}{দরকারী ফাইল}
	\begin{alertblock}{আবশ্যিক ফাইল}
	\begin{itemize}
		\item {\rm beamerthemexelatexbengali.sty}
	\end{itemize}
	\end{alertblock}

	\begin{block}{জিলাটেক বেঙ্গলি}
	\begin{itemize}
		\item {\rm xelatexbengali.sty} ফাইলছাড়া সবগুলো এখানেও দরকার।
	\end{itemize}
	\end{block}

	\begin{block}{{\rm \LaTeX} কম্পাইলার}
	\begin{itemize}
		\item কেবলমাত্র {\rm \XeLaTeX} জিলাটেক
	\end{itemize}
	\end{block}

	\begin{exampleblock}{অপশনাল ফাইল}
	\begin{itemize}
		\item {\rm sample-slide.tex}
	\end{itemize}
	\end{exampleblock}

\end{frame}

\section{অপশন সমুহ}

\begin{frame}{অপশন সমুহ}
	\begin{small}	
	\begin{enumerate}
		\item \rm{\structure{Somber:} if true, use dull colors; else bright colors.}
		\item \rm{\structure{Ruled:} if true, draws a rule below the frame title.} 
		\item \rm{\structure{Square:} if true, use squares for enumerate; else circles.}
		\item \rm{\structure{Arrow:} if true, use arrows for \alert{itemize}; else follow \alert{enumerate}.}
		\item \rm{\structure{Shadow:} if true, shadows for 3D appearances of the blocks.}
	\end{enumerate}
	\end{small}
\end{frame}

\begin{frame}{এনভায়রনমেন্টস}
	‌\begin{theorem}[জোড় সংখ্যা]
	জোড় সংখ্যা মাত্রই দুই দিয়ে বিভাজ্য।
	\end{theorem}
	‌\begin{lemma}[বিজোড় সংখ্যা]
	বিজোড় সংখ্যা মাত্রই দুই দিয়ে বিভাজ্য নয়।
	\end{lemma}
	‌\begin{corollary}[স্বাভাবিক সংখ্যা]
	জোড় ও বিজোড় সবগুলো মিলিয়ে স্বাভাবিক সংখ্যা।
	\end{corollary}
\end{frame}

\begin{frame}{এনভায়রনমেন্টস}
	‌\begin{problem}[বৃত্ত অংকন]
	তিনটি বিন্দু দিয়ে একটি বৃত্ত আঁকুন।
	\end{problem}
	‌\begin{definition}[শুন্য]
	শুন্য মানে কিছুই না।
	\end{definition}
	‌\begin{example}[এক]
	১ বা এক বা {\rm I} বা {\rm 1}.
	\end{example}
\end{frame}

\begin{frame}{এনভায়রনমেন্টস ‌\ldots}
	‌\begin{solution}[বৃত্ত অংকন]
	দুটো করে বিন্দু নিয়ে দুটো রেখা এঁকে তাদের মধ্যবিন্দুতে লম্ব আঁকুন, লম্বদ্বয়ের ছেদবিন্দুকে কেন্দ্র করে যেকোন একটি বিন্দু পর্যন্ত ব্যাসার্ধ নিয়ে বৃত্ত আঁকুন।
	\end{solution}
	‌\begin{proof}[বৃত্ত অংকন]
	মধ্যবিন্দুতে লম্বগুলো সংশ্লিষ্টবিন্দু গুলো থেকে সমদুরত্বে। লম্বদ্বয়ের ছেদবিন্দু তিনটি বিন্দু থেকেই সমদুরত্বে।
	\end{proof}
	‌\begin{exercise}[মধ্যবিন্দু]
	দুটো বিন্দুর সংযোগ রেখার মধ্যবিন্দু বের করুন।
	\end{exercise}
\end{frame}

\begin{frame}{এনভায়রনমেন্টস ‌\ldots}
	\begin{proposition}[সত্য]
	নহে সত্য নহে মিথ্যায় নহে সত্য-মিথ্যা। 
	\end{proposition}
	\begin{equation}
	\mathbn{a + b = c}
	\end{equation}
	\begin{figure}
	‌{‌\Huge\bnem ‌নকশা}
	\caption{নকশার উদাহরণ}
	\end{figure}
	\vspace{-3em}
	\begin{table}
	\caption{সারণীর উদাহরণ}
	‌\begin{tabular}{|c|c|}
	\hline
	অ & আ‌‌\\\hline
	ক & খ\\\hline
	\end{tabular}
	\end{table}
\end{frame}

\begin{frame}{উপসংহার}
যে কোন ভুলভ্রান্তি ও পরামর্শ আমাদের জানাতে অনুরোধ করছি। আমরা সংশোধন ও পরিবর্ধনের চেষ্টা করব। জানা সমস্যাগুলোর মধ্যে {\rm bibtex} এর সাথে সমন্বয় এখনো করা হয় নাই। কাজেই তথ্যসুত্র গুলোর রেফারেন্সে ইংলিশ অক্ষর চলে আসতে পারে। বাংলায় প্রেজেন্টশন বানানো উপভোগ করুন। অন্যদের জানিয়ে সেই আনন্দ ছড়িয়ে দিন।
\end{frame}
\end{document}

